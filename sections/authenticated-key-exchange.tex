\section{Authenticated Key Exchange}
\label{sec:AKE}

In practice, our key exchange protocol should provide authentication.
This means that after the key exchange, both parties should be sure that their communication partner is indeed the intended party and that no other party has knowledge of their shared symmetric key.

\paragraph{Key Exchange with Interaction} Consider the \emph{key exchange with interaction}~(IKE) protocol from the exercise. 
As a reminder, the protocol (here called two-pass key exchange) is defined as follows:

A two-pass key exchange $\TKE=(\TKEsnd,\TKErsp,\TKErcv)$ is a tuple of three algorithms with state space~$\stsp$, initial ciphertext space~$\csp_I$, response ciphertext space~$\csp_R$, and symmetric key space~$\ksp$:

\begin{itemize}
    \item $\TKEsnd: \emptyset \tor \stsp\times\csp_I$
    \item $\TKErsp: \csp_I \tor \ksp\times\csp_R$
    \item $\TKErcv: \stsp\times\csp_R \to \ksp$
\end{itemize}

A normal protocol execution can be seen in Figure~\ref{fig:tke:overview}.

The problem with a two-pass key exchange is that it is vulnerable to a \emph{Man-in-the-Middle} attack, where the adversary $\advA$ intercepts the messages between Alice and Bob and replaces them with its own messages (and therefore its own keys)~(Figure~\ref{fig:tke:MitM}).

\begin{figure}[!ht]
    \centering
    \begin{subfigure}{.49\textwidth}
        \centering
        % !TeX root = ..\..\main.tex
\begin{tabular}{cC{1cm}c}
    \textbf{Alice} &  & \textbf{Bob}\\
    & $\xlongrightarrow{c_A}$ & \\
    $\downarrow$& $\xlongleftarrow{c_B}$ & $\downarrow$\\
    $k$ & & $k$
\end{tabular}
        \caption{$\TKE$ protocol execution.}
        \label{fig:tke:overview}
    \end{subfigure}\hfill
    \begin{subfigure}{.49\textwidth}
        \centering
        % !TeX root = ..\..\main.tex
\begin{tabular}{cC{1cm}r|lC{1cm}c}
    \textbf{Alice} &  & \multicolumn{2}{c}{$\advA$} & & \textbf{Bob}\\
    & $\xlongrightarrow{c_A}$ &  &  & $\xlongrightarrow{c_A'}$ & \\
    $\downarrow$ & $\xlongleftarrow{c_B'}$ & & & $\xlongleftarrow{c_B}$ & $\downarrow$\\
    $k_1$ & & $k_1$ & $k_2$ & & $k_2$ \\
\end{tabular}
        \caption{$\TKE$ Man-in-the-Middle attack.}
        \label{fig:tke:MitM}
    \end{subfigure}
    \caption{Two-pass key exchange protocol.}
\end{figure}

\paragraph{Key Exchange with Authentication} To prevent this and to provide actual authentication, we use \emph{Authenticated Key Exchange}~(AKE) protocols.
An authenticated key exchange $\AKE=(\AKEgen,\allowbreak\AKEsnd,\allowbreak\AKErsp_R,\allowbreak\AKErsp_I,\allowbreak\AKErcv)$ is a tuple of five algorithms with secret key space~$\sksp$, public key space~$\pksp$, state space~$\stsp$, ciphertext space~$\csp$, and symmetric key space~$\ksp$:
\begin{itemize}
    \item $\AKEgen: \emptyset \tor \sksp\times\pksp$
    \item $\AKEsnd: \sksp\times\pksp \tor \stsp\times\csp$
    \item $\AKErsp_R: \sksp\times\csp \tor \stsp\times\csp$
    \item $\AKErsp_I: \stsp\times\csp \tor \ksp\times\csp$
    \item $\AKErcv: \sksp\times\stsp\times\csp \to \pksp\times\ksp$
\end{itemize}

An overview of normal protocol execution is given in Figure~\ref{fig:ake:overview}. 

\begin{figure}[!ht]
    \centering
    % !TeX root = ..\..\main.tex
\begin{tabular}{lC{3cm}l}
    \textbf{Alice} &  & \textbf{Bob}\\
    $(\sk_A, \pk_A)\getsr\AKEgen$ & $\xleftrightarrow{\pk_A,\pk_B}$ & $(\sk_B, \pk_B)\getsr\AKEgen$ \\
    $(\st, c_1)\getsr\AKEsnd(\sk_A, \pk_B)$ & $\xlongrightarrow{c_1}$ & \\
    & $\xlongleftarrow{c_2}$ & $(\st, c_2)\getsr\AKErsp_R(\sk_B, c_1)$\\
    $(k,c_3)\getsr\AKErsp_I(\st, c_2)$ & $\xlongrightarrow{c_3}$ & \\
    & & $(k, \pk_A)\gets\AKErcv(\st, c_3)$
\end{tabular}
    \caption{Authenticated key exchange protocol.}
    \label{fig:ake:overview}
\end{figure}

\paragraph{Correctness} An $\AKE$ protocol is correct if Alice and Bob compute the same key $k$ for every honest protocol execution.

\paragraph{Security} The adversary has access to the following oracles with the goal of using the $\Ochall$ oracle to output bit $b$.

\setlength\itemizeskip{3.5cm}
\begin{itemize}
    \item $\Ogen() \to \pk$:\hfill\makebox[\linewidth-\itemizeskip][l]{Generate new party}
    \item $\Osnd(i, s, \pk) \to c$:\hfill\makebox[\linewidth-\itemizeskip][l]{Lets instance $s$ of party $i$ start a session with party $\pk$}
    \item $\Orsp(i,s,c)\to c'$:\hfill\parbox[t][][l]{\linewidth-\itemizeskip}{Lets instance $s$ of party $i$ respond to $c$\\Also used for final receipt}
    \item $\Ocorrupt(i)\to\sk_i$:\hfill\makebox[\linewidth-\itemizeskip][l]{Corrupts party $i$'s long-term secret key}
    \item $\Oreveal(i,s)\to k_{i,s}$:\hfill\makebox[\linewidth-\itemizeskip][l]{Reveals final session key computed by instance $s$ of party $i$} 
    \item $\Ochall(i,s)\to k_{i,s,b}$:\hfill\makebox[\linewidth-\itemizeskip][l]{Either outputs real or random session key of instance $c$ of party $i$}
\end{itemize}

\paragraph{Trivial winning conditions} The trivial attacks are listed in Figure~\ref{fig:ake:trivial}. 
Note that the names of the trivial attacks give the general strategy and that the provided pseudocode can sometimes be changed (e.g., by switching the order of $\Ochall$ and $\Oreveal$ in the first attack).
The third attack in particular requires only that $\pk_j$ is the intended partner of $(i,s)$ and that $(i,s)$ has no session partners.

\begin{figure}[!ht]
    \centering
    % !TeX root = ..\..\main.tex
\begin{minipage}[t]{.5\textwidth}
    Attack 1: $\Ochall(i,s)$ and $\Oreveal(i,s)$:
    \begin{enumerate}[topsep=\smallskipamount]
    \item Choose $(i,s)$ fittingly.
    \item $\Ochall(i,s)\to k_{i,s}$
    \item $\Oreveal(i,s)\to k_{i,s}'$
    \item Stop with $k_{i,s}\not= k_{i,s}'$
    \end{enumerate}
    \medskip
    Attack 2: $\Ochall(i,s)$ and $\Oreveal(j,t)$:
    \begin{enumerate}[topsep=\smallskipamount]
    \item Choose $(i,s)$ fittingly.
    \item $\Ochall(i,s)\to k_{i,s}$
    \item Let $(j,t)$ be a \emph{session partner} of $(i,s)$.
    \item $\Oreveal(j,t)\to k_{j,t}$
    \item Stop with $k_{i,s}\not= k_{j,t}$
    \end{enumerate}
\end{minipage}\hfill
\begin{minipage}[t]{.5\textwidth}
    Attack 3: $\Ochall(i,s)$ and $\Ocorrupt(j)$:
    \begin{enumerate}[topsep=\smallskipamount]
    \item $\Ogen\to\pk_i$
    \item $\Ogen\to\pk_j$
    \item $\Osnd(i,s,\pk_j)\to c_1$
    \item $\Ocorrupt(j)\to\sk_j$
    \item $\AKErsp_R(\sk_j,c_1)\tor(\st,c_2)$
    \item $\Orsp(i,s,c_2)\to c_3$
    \item $\AKErcv(\st, c_3)\to (k,\pk_i)$
    \item $\Ochall(i,s)\to k'$
    \item Stop with $k\not= k'$
    \end{enumerate}
\end{minipage}

    \caption{Trivial attacks on authenticated key exchange.}
    \label{fig:ake:trivial}
\end{figure}

\paragraph{Session partners} When talking about $\AKE$ it is helpful to formalize the concept of \emph{session partners}. 
In literature however, there are different definitions of session partners, which can be summarized as follows:
\begin{itemize}
    \item \emph{Matching Conversations} \cite{C:BelRog93}: ``Both instances saw the same transcript''
    \item \emph{Session identifiers}: ``Both instances saw the same relevant parts of the transcript''
    \item \emph{Original Key Partnering} \cite{EPRINT:LiSch17}: ``Using the same randomness, $(i,s)$ and $(j,t)$ would have computed the same key in an honest execution''
\end{itemize}
This lecture uses the \emph{Matching Conversations} definition when talking about session partners.

\paragraph{Constructions} We present two constructions of authenticated key exchange protocols. 
The first one (Figure~\ref{fig:ake:signed_dh}) is based on the Diffie-Hellman key exchange and uses signatures to provide integrity.

The second AKE construction (Figure~\ref{fig:ake:kem}) uses a KEM and a multi-key PRF (Figure~\ref{fig:prf:multi_key:ind}). 
In Figure~\ref{fig:ake:kem}, the secrets that can be learned by the adversary $\advA$ after corrupting Alice and Bob are colored orange and blue, respectively.
For example, after corrupting Bob and learning the secret key $\sk_B$, $\advA$ can decapsulate $c_1$ (which can be learned by eavesdropping on the communication) and obtain $k_1$.
The symmetric key $k$ will still remain secure, since $k_3$ will not be known to $\advA$, even if both Alice and Bob were corrupted.
The reason for this is the usage of fresh keys $\sk',\pk'$ for each new execution.

\begin{figure}[!ht]
    \centering
    % !TeX root = ..\..\main.tex
\begin{tabular}{lC{3cm}l}
    \textbf{Alice} &  & \textbf{Bob}\\
    $a\getsr\ZZ_p^*$ & $\xlongrightarrow{g^a}$ & $b\getsr\ZZ_p^*$ \\
     & $\xlongleftarrow{g^b,\sigma_B}$ & $\sigma_B\gets\SIGsig(\sk_B,(g^a, g^b))$ \\
     $\sigma_A\getsr\SIGsig(\sk_A,(g^a, g^b, \sigma_B))$ & $\xlongrightarrow{\sigma_A}$ & \\
\end{tabular}
    \caption{$\AKE$ based on Diffie-Hellman key exchange and signature scheme $\SIG$.}
    \label{fig:ake:signed_dh}
\end{figure}

\begin{figure}[!ht]
    \centering
    % !TeX root = ..\..\main.tex
\begin{tabular}{lC{3cm}l}
    \textbf{Alice}$(\sk_A, \pk_A)$ &  & \textbf{Bob}$(\sk_B, \pk_B)$\\
    $(k_1,c_1)\getsr\KEMenc(\pk_B)$ & &\\
    $(\sk',\pk')\getsr\AKEgen$ & $\xrightrightarrow{c_1,\pk'}$ & $k_1\gets\KEMdec(sk_B,c_1)$ \\
    & & $(k_2,c_2)\getsr\KEMenc(\pk_A)$\\
    & & $(k_3,c_3)\getsr\KEMenc(\pk')$\\
    & $\xlongleftarrow{c_2,c_3,\tau_B}$ & $(\tau_B, k_B)\gets\PRFf(k_1,\mathit{tag}=(c_1,\pk',c_2,c_3))$\\
    $(k,\tau_A)\gets\PRFf(k_)
\end{tabular}
    \caption{$\AKE$ based on an $\INDCCA$ secure $\KEM$ and $\IND$ secure $\PRF$. Colored secrets are known to $\advA$ after corruption of \textcolor{orange}{Alice} and \textcolor{blue}{Bob}.}
    \label{fig:ake:kem}
\end{figure}

\subsection{Real-World Authenticated Key Exchange}
Even though there are many more real world examples, this script will focus on three examples:

\subsubsection{Transport Layer Security~(TLS)}
TLS key exchange is similar to the signed Diffie-Hellman construction in Figure~\ref{fig:ake:signed_dh}.
An overview of the protocol is given in Figure~\ref{fig:ake:tls}.

\begin{figure}[!ht]
    \centering
    % !TeX root = ..\..\main.tex
\begin{tabular}{lC{3cm}l}
    \textbf{Alice} &  & \textbf{Bob}\\
    $x\getsr\ZZ_p^*$ & $\xlongrightarrow{g^x}$ & $y\getsr\ZZ_p^*$ \\
    & $\xlongleftarrow{g^y,\sigma}$ & $\sigma\gets\SIGsig(\sk_B,(g^x, g^y))$ \\
    & $\xlongrightarrow[\text{with }g^{xy}]{\text{Key Confirmation}}$ & \\
    \parbox[b][][t]{2.5cm}{\centering$\downarrow$\\Key derivation} &  & \parbox[b][][t]{2.5cm}{\centering$\downarrow$\\Key derivation}
\end{tabular}
    \caption{TLS key exchange.}
    \label{fig:ake:tls}
\end{figure}

\subsubsection{Noise Framework}
The Noise Framework (\url{https://noiseprotocol.org/noise.html}) is a set of rules to combine Diffie-Hellman key exchange with Authenticated Encryption~(AE).
The framework is very flexible, especially in a pre-configure context.
It is used by WhatsApp for its client-to-server communication and by WireGuard among others.

One protocol defined in the Noise Framework is ``NK''. 
Following the naming convention of Noise protocols, the ``N'' signifies that the initiator has \textbf{n}o static key and the ``K'' signifies that the responder has a static key \textbf{k}nown to the initiator.

In the documentation of the Noise Framework, the protocols are described in a format similar to the one in Figure~\ref{fig:ake:noise:nk}.
Figure~\ref{fig:ake:noise:nk_overview} gives an overview of the protocol execution in the format used in this script.
The $s$ in the figure is the static key of the responder, which is transmitted to the initiator before the protocol begins (shown by the dots afterwards).
An $e$ is an \emph{ephemeral} key (short-lived key) that correspond to a DH public key-share (e.g. $g^x$).
If two letters are combined a combination of the two keys is formed, e.g. $es$ and $g^{x\sk_B}$.

The messages sent during and after the protocol execution offer different levels of security: 
The encryption of $m_1$ offers forward security for the sender but not for the responder, while authenticity is only guaranteed for the responder.
Anonymity is guaranteed for both parties, but for the responder this holds only as long as he is not corrupted.
For $m_i, i\geq 2$ the encryption offers forward security for both parties, while the other security guarantees remain unchanged.

\begin{figure}[!ht]
    \centering
    \begin{subfigure}{.33\textwidth}
        \centering
        % !TeX root = ..\..\main.tex
\begin{tabular}{cl}
    $\leftarrow$ & s\\
    \dots & \\
    $\rightarrow$ & e, es\\
    $\leftarrow$ & e, ee
\end{tabular}
        \caption{Specification of the ``NK'' protocol in the Noise Framework format.}
        \label{fig:ake:noise:nk}
    \end{subfigure}\hfill
    \begin{subfigure}{.66\textwidth}
        \centering
        % !TeX root = ..\..\main.tex
\begin{tabular}{lC{3cm}l}
    \textbf{Alice} & & \textbf{Bob} \\
    $x\getsr\ZZ_p^*$ & & \\
    $k_1\gets\PRFf(\pk_B^x)$ & $\xlongrightarrow{g^x,\SYEenc(k_1,m_1)}$ & $k_1\gets\PRFf((g^x)^{\sk_B})$\\
    & & $y\getsr\ZZ_p^*$ \\
    & & $k_2\gets\PRFf(k_1, g^{xy})$\\
    & $\xlongleftarrow{\substack{\SYEenc(k_1,g^y)\\\SYEenc(k_2,m_2)}}$ & \\
    & $\xleftrightarrow{\SYEenc(k_2,m_i)}$ &
\end{tabular}
        \caption{Overview of ``NK'' protocol execution.}
        \label{fig:ake:noise:nk_overview}
    \end{subfigure}
    \caption{Noise protocol ``NK''.}
    \label{fig:ake:noise}
\end{figure}

\subsubsection{X3DH}
X3DH is an extended and authenticated 2-pass key exchange. 
It is used in the Ratcheted Key Exchange protocol~(Section \ref{sec:rke}).

The X3DH protocol is designed for asynchronous settings, where two parties want to establish a shared secret key, but one of them might be offline.
For this, one party (here named Bob) regularly sends Diffie-Hellman public keys to the server.
His long term key~(LTK), a prekey and a set of one-time keys. 
The prekey and the one-time keys are renewed regularly, but the one-time keys are renewed more often.

If Alice wants to then establish a public key with Bob, she can fetch Bobs keys from the server.
She also generates an ephemeral Diffie-Hellman exponent, which is then combined with Bobs keys and used to derive the symmetric key through a PRF. (Figure~\ref{fig:ake:x3dh})

If all of Bobs secret keys were used, and he has not renewed them yet, the key $g^{ab}$ is left out.
This reduces the forward security of the protocol, since the prekey is renewed much less often.

The X3DH protocol guarantees strong forward security, authenticity and deniability (both parties can deny their participation).

\begin{figure}[!ht]
    \centering
    \input{figures/ake/ake_X3DH}
    \caption{X3DH key exchange.}
    \label{fig:ake:x3dh}
\end{figure}
