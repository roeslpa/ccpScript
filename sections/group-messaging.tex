\section{Messaging in Groups}

Until now, we have only discussed messaging in two party settings.
However, most messengers also support group messaging.

\paragraph{Pairwise Channels} 
One method to implement group messaging is to use existing pairwise channels (e.g., Double Ratchet channels).
This method was used in early version of Signal.
The advantage is, that forward and post-compromise security are inherited from the pairwise channels.
The problem is that this introduces a communication overhead of $n-1$: 
If one wants to send a message in a group with $n$ group members, $n-1$ different encryptions have to be constructed and sent through the pairwise channels.

Another problem is managing the membership. 
Earlier, this was done with a shared group secret as proof of membership.
This introduces new problems, as the group secret has to be updated on every membership change.
As a result, current implementations often use sophisticated server-assisted membership protocols.

\paragraph{Sender Key Mechanism}
This method is used in WhatsApp, Signal and many others.
The idea is, that each member maintains $n$ sender ratchets per group, one for each group member.
Then, the sender only has to create one ciphertext for the group, which is then sent to all group members.
Using this method, the cryptographic overhead in ciphertext is eliminated, but there is no more post-compromise security.
Forward security is still provided immediately, but to achieve post-compromise security, the sender keys have to be renewed periodically via pairwise channels, which reintroduces cryptographic overhead.

\paragraph{Message Layer Security Standard}
A third method of managing group messaging is the Message Layer Security~(MLS) standard~\cite{RFCMLS}.
MLS uses tree based structures create and manage a shared group secret even if the users are offline most of the time.
These structures will be explained in more detail in the following section.

\subsection{Asynchronous Ratcheting Tree~(ART)}
ART \cite{EPRINT:CCGMM17} uses Diffie-Hellman keys to asynchronously create a shared group secret.
A tree for eight group members can be seen in Figure~\ref{fig:art:art}.
The eight parties are at the leave nodes of the tree.

\begin{figure}[!ht]
    \centering
    % !TeX root = ..\..\main.tex
\begin{tikzpicture}[%
    level 1/.style={sibling distance=8cm}, %distances between nodes on level 1
    level 2/.style={sibling distance=4cm},
    level 3/.style={sibling distance=2cm},
]
    %draw the tree
    \scoped [align=center] %allow all nodes to include line breaks (\\)
        \node (n) {$\pk_{AH}\gets g^{k_{AH}}$\\$k_{AH}\gets g^{k_{AD}k_{EH}}$} %set name of root node to n --> childs are named n-1/n-2 and grandchilds are n-1-1/n-1-2...
            child {node {$\pk_{AD}\gets g^{k_{AD}}$\\$k_{AD}\gets (\pk_{AB})^{k_{CD}}=(\pk_{CD})^{k_{AB}}$}
                child {node {$\pk_{AB}\gets g^{k_{AB}}$\\$k_{AB}\gets g^{ab}$}
                    child {node {Alice\\$a, g^a$}}
                    child {node {Bob\\$b, g^b$}}
                }
                child {node {$\pk_{CD}\gets g^{k_{CD}}$\\$k_{CD}\gets g^{cd}$}
                    child {node {Charlie\\$c, g^c$}}
                    child {node {Dave\\$d, g^d$}}
                }}
            child {node {$\pk_{EH}$\\$k_{EH}$}
                child {node {$\pk_{EF}$\\$k_{EF}$}
                    child {node {Eve\\$e, g^e$}}
                    child {node {\dots} edge from parent[draw=none]} %draw the dots to indicate that there are more nodes; dont draw the edge from parent
                }
                child[missing]
            };
    
    %ARROWS
    %transform canvas is used to shift the upper arrow up and the lower arrow down

    %between Alice and Bob
    \draw[->, transform canvas={yshift= 2}] (n-1-1-1) -- (n-1-1-2) node[midway, above] {$g^a$};
    \draw[<-, transform canvas={yshift=-2}] (n-1-1-1) -- (n-1-1-2) node[midway, below] {$g^b$};
    %between Charlie and Dave
    \draw[->, transform canvas={yshift= 2}] (n-1-2-1) -- (n-1-2-2) node[midway, above] {$g^c$};
    \draw[<-, transform canvas={yshift=-2}] (n-1-2-1) -- (n-1-2-2) node[midway, below] {$g^d$};
    %between AB and CD
    %If the arrow is drawn directly from node n-1-1 to n-1-2, it will be way to long and look bad
    %--> place nodes left and right of n-1-1 and n-1-2 and use them as anchors for the arrows
    \node (n-1-1ArrowAnchor) [right=1mm of n-1-1] {};
    \node (n-1-2ArrowAnchor) [left=1mm of n-1-2] {};
    \draw[->, transform canvas={yshift= 2}] (n-1-1ArrowAnchor) -- (n-1-2ArrowAnchor) node[midway, above] {$\pk_{AB}$};
    \draw[<-, transform canvas={yshift=-2}] (n-1-1ArrowAnchor) -- (n-1-2ArrowAnchor) node[midway, below] {$\pk_{CD}$};
    %between AD and EH
    \node (n-1ArrowAnchor) [right=1cm of n-1] {};
    \node (n-2ArrowAnchor) [left=2cm of n-2] {};
    \draw[->, transform canvas={yshift= 2}] (n-1ArrowAnchor) -- (n-2ArrowAnchor) node[midway, above] {$\pk_{AD}$};
    \draw[<-, transform canvas={yshift=-2}] (n-1ArrowAnchor) -- (n-2ArrowAnchor) node[midway, below] {$\pk_{EH}$};
\end{tikzpicture}
    \caption{An example of an Asynchronous Ratcheting Tree~(ART) with eight group members.}
    \label{fig:art:art}
\end{figure}

Now assume that Charlie was compromised at some point and now wants to update the group keys.
Charlie can sample a new Diffie-Hellman share $c'$ and propagate the change upwards through the tree as seen in Figure~\ref{fig:art:art_update}.
It is important to note, that the other parties do not have to be online for Charlie to update the keys.
Once they are online, they can recompute the new group key from the new public keys.

\begin{figure}[!ht]
    \centering
    % !TeX root = ..\..\main.tex
\begin{tikzpicture}[%
    level 1/.style={sibling distance=8cm}, %distances between nodes on level 1
    level 2/.style={sibling distance=4cm},
    level 3/.style={sibling distance=2cm},
]
    %draw the tree
    \scoped [align=center] %allow all nodes to include line breaks (\\)
        \node (n) {7. $\pk'_{AH}\gets g^{k'_{AH}}$\\$k'_{AH}\gets g^{k'_{AD}k_{EH}}$} %set name of root node to n --> childs are named n-1/n-2 and grandchilds are n-1-1/n-1-2...
            child {node {5. $\pk'_{AD}\gets g^{k'_{AD}}$\\$k'_{AD}\gets (\pk_{AB})^{k'_{CD}}=(\pk'_{CD})^{k_{AB}}$}
                child {node {$\pk_{AB}\gets g^{k_{AB}}$\\$k_{AB}\gets g^{ab}$}
                    child {node {Alice\\$a, g^a$}}
                    child {node {Bob\\$b, g^b$}}
                }
                child {node {3. $\pk'_{CD}\gets g^{k'_{CD}}$\\$k'_{CD}\gets g^{c'd}$}
                    child {node {1. Charlie\\$c', g^{c'}$}}
                    child {node {Dave\\$d, g^d$}}
                }}
            child {node {$\pk_{EH}$\\$k_{EH}$}
                child {node {$\pk_{EF}$\\$k_{EF}$}
                    child {node {Eve\\$e, g^e$}}
                    child {node {\dots} edge from parent[draw=none]} %draw the dots to indicate that there are more nodes; dont draw the edge from parent
                }
                child[missing]
            };
    
    %ARROWS
    %transform canvas is used to shift the upper arrow up and the lower arrow down

    %between Alice and Bob
    \draw[->, transform canvas={yshift= 2}] (n-1-1-1) -- (n-1-1-2) node[midway, above] {$g^a$};
    \draw[<-, transform canvas={yshift=-2}] (n-1-1-1) -- (n-1-1-2) node[midway, below] {$g^b$};
    %between Charlie and Dave
    \draw[->, transform canvas={yshift= 2}] (n-1-2-1) -- (n-1-2-2) node[midway, above] {2. $g^{c'}$};
    \draw[<-, transform canvas={yshift=-2}] (n-1-2-1) -- (n-1-2-2) node[midway, below] {$g^d$};
    %between AB and CD
    %If the arrow is drawn directly from node n-1-1 to n-1-2, it will be way to long and look bad
    %--> place nodes left and right of n-1-1 and n-1-2 and use them as anchors for the arrows
    \node (n-1-1ArrowAnchor) [right=1mm of n-1-1] {};
    \node (n-1-2ArrowAnchor) [left=1mm of n-1-2] {};
    \draw[->, transform canvas={yshift= 2}] (n-1-1ArrowAnchor) -- (n-1-2ArrowAnchor) node[midway, above] {$\pk_{AB}$};
    \draw[<-, transform canvas={yshift=-2}] (n-1-1ArrowAnchor) -- (n-1-2ArrowAnchor) node[midway, below] {4. $\pk'_{CD}$};
    %between AD and EH
    \node (n-1ArrowAnchor) [right=1cm of n-1] {};
    \node (n-2ArrowAnchor) [left=2cm of n-2] {};
    \draw[->, transform canvas={yshift= 2}] (n-1ArrowAnchor) -- (n-2ArrowAnchor) node[midway, above] {$6. \pk'_{AD}$};
    \draw[<-, transform canvas={yshift=-2}] (n-1ArrowAnchor) -- (n-2ArrowAnchor) node[midway, below] {$\pk_{EH}$};
\end{tikzpicture}
    \caption{Steps of Charlie updating their key in the ART of Figure~\ref{fig:art:art}.}
    \label{fig:art:art_update}
\end{figure}

ART has a cryptographic overhead of $O(\log n)$, for updating group keys in a group of $n$ members.
The group membership is also dynamic: When a new member joins, the tree is extended and when a member leaves, the tree is rewritten and the removed users path is updated.
Since ART relies on DHKE, it is however not post-quantum secure.

\subsection{Tree KEM}
Instead of using DHKE, one can also use a key encapsulation mechanism~(KEM) to create a shared group key.

\begin{figure}[!ht]
    \centering
    % !TeX root = ..\..\main.tex
\newcommand{\gen}{\mathrm{gen}}
\newcommand{\enc}{\mathrm{enc}}
\begin{tikzpicture}[%
    level 1/.style={sibling distance=8cm}, %distances between nodes on level 1
    level 2/.style={sibling distance=4cm},
    level 3/.style={sibling distance=2cm},
]
    %draw the tree
    \scoped [align=center] %allow all nodes to include line breaks (\\)
        \node (n) {$\pk_{AH}\gets\gen(k_{AH})$\\$(k_{AH}, c_{AH})\getsr\enc(\pk_{EH})$} %set name of root node to n --> childs are named n-1/n-2 and grandchilds are n-1-1/n-1-2...
            child {node {$\pk_{AD}\gets\gen(k_{AD})$\\$(k_{AD}, c_{AD})\getsr\enc(\pk_{CD})$}
                child {node {$\pk_{AB}\gets\gen(k_{AB})$\\$(k_{AB}, c_{AB})\getsr\enc(\pk_B)$}
                    child {node {Alice\\$\sk_A, \pk_A$}}
                    child {node {Bob\\$\sk_B, \pk_B$}}
                }
                child {node {$\pk_{CD}\gets\gen(k_{CD})$\\$(k_{CD}, c_{CD})\getsr\enc(\pk_D)$}
                    child {node {Charlie\\$\sk_C, \pk_C$}}
                    child {node {Dave\\$\sk_D, \pk_D$}}
                }}
            child {node {$\pk_{EH}$\\$k_{EH}$}
                child {node {$\pk_{EF}$\\$k_{EF}$}
                    child {node {Eve\\$\sk_E, \pk_E$}}
                    child {node {\dots} edge from parent[draw=none]} %draw the dots to indicate that there are more nodes; dont draw the edge from parent
                }
                child[missing]
            };
    
    %ARROWS
    %between Alice and Bob
    \draw[->] (n-1-1-1) -- (n-1-1-2) node[midway, above] {$c_{AB}$};
    %between Charlie and Dave
    \draw[->] (n-1-2-1) -- (n-1-2-2) node[midway, above] {$c_{CD}$};
    %between AB and CD
    \draw[->] (n-1-1) -- (n-1-2) node[midway, above] {$c_{AD}$};
    %between AD and EH
    %If the arrow is drawn directly from node n-1-1 to n-1-2, it will be way to long and look bad
    %--> place nodes left and right of n-1-1 and n-1-2 and use them as anchors for the arrows
    % \node (n-1-1ArrowAnchor) [right=1mm of n-1-1] {};
    % \node (n-1-2ArrowAnchor) [left=1mm of n-1-2] {};
    \node (n-1ArrowAnchor) [right=1cm of n-1] {};
    \node (n-2ArrowAnchor) [left=2cm of n-2] {};
    \draw[->] (n-1ArrowAnchor) -- (n-2ArrowAnchor) node[midway, above] {$c_{AH}$};
\end{tikzpicture}
    \caption{An example of a Tree KEM with eight group members. $\mathrm{gen}(k)$ is the KEM gen algorithm with randomness $k$.}
    \label{fig:art:art_kem}
\end{figure}

If Bob wants to update the group key, he does the following operations:
\begin{enumerate}
    \item $(\sk'_B, \pk'_B)\getsr\mathrm{gen}$
    \item $(k'_{AB}, c'_{AB})\getsr\mathrm{enc}(\pk_A)$\\
        $(s_\pk, s_{\mathrm{enc}})\gets\KDF(k'_{AB})$\\
        $\pk_{AB}\gets\mathrm{gen}(s_\pk)$
    \item $(k'_{AD}, c'_{AD})\gets\mathrm{enc}(\pk_{CD}; s_{\mathrm{enc}})$\\
        $(s_\pk, s_{\mathrm{enc}})\gets\KDF(k'_{AD})$\\
        $\pk'_{AD}\gets\mathrm{gen}(s_\pk)$
    \item $(k'_{AH}, c'_{AH})\gets\mathrm{enc}(\pk_{EH}; s_{\mathrm{enc}})$\\
        $(s_\pk, k_{\mathit{group}})\gets\KDF(k'_{AH})$\\
        $\pk'_{AH}\gets\mathrm{gen}(s_\pk)$
\end{enumerate}

% We need to use seeds $s_\pk$ and $s_{\mathrm{enc}}$ as otherwise 