\documentclass[a4paper,orivec]{llncs}

\newcommand{\alert}[1]{{\color{red}#1}}

\usepackage{fullpage}
\pagestyle{plain}
\setcounter{tocdepth}{2}

%!TEX root=main.tex

\usepackage{lmodern}
\usepackage[T1]{fontenc}
\usepackage[utf8]{inputenc}

\usepackage{amsfonts,amsmath,amssymb,mathtools}

\usepackage{graphicx}

\usepackage{array}

\usepackage{color,xcolor}
\definecolor{gray}{gray}{0.5}
\definecolor{darkblue}{rgb}{0,0,0.5}
\definecolor{darkgreen}{rgb}{0,0.5,0}
\usepackage[colorlinks=true,linkcolor=darkblue,urlcolor=darkblue,citecolor=darkgreen,pdftitle={Cryptographic Communication Protocols: Key Exchange and Channels}]{hyperref}

\usepackage[strings]{underscore}

\newcommand{\codescalefactor}{0.8}
\newcommand{\tikzscalefactor}{0.8}

\usepackage{nicodemus}
\usepackage{bpmarker}

%draw dashed boxes
\usepackage{dashbox}
\setlength{\dashlength}{4pt}
\setlength{\dashdash}{2pt}

%tikz and libraries
\usepackage{tikz}
\usetikzlibrary{arrows.meta, positioning, calc, trees, shapes, intersections, tikzmark}

%some symbols (e.g. lightning bolt)
\usepackage{marvosym}

%placing correlated figures next to each other
\usepackage{subcaption}
%!TEX root=main.tex

\let\oldparagraph=\paragraph
\renewcommand\paragraph[1]{\oldparagraph{#1.}}

\newcolumntype{C}[1]{>{\centering\arraybackslash\hspace{0pt}}p{#1}}

\newcommand{\algbox}[2]{\fbox{\parbox{#1}{#2}}}
\newcommand{\arrbox}[2]{\parbox{#1}{\centering#2}}

\newcommand{\getsr}{\gets_\$}
\newcommand{\tor}{\to_\$}
\newcommand{\es}{\epsilon}
\newcommand{\T}{\mathtt{tru}}
\newcommand{\F}{\mathtt{fal}}
\newcommand{\secp}{\kappa}

\newcommand{\ZZ}{\mathbb{Z}}
\newcommand{\BB}{\{0,1\}}
\newcommand{\Bool}{\{\T,\F\}}
\newcommand{\NN}{\mathbb{N}}
\newcommand{\PS}{\mathcal{P}}

\newcommand{\advA}{\mathcal{A}}
\newcommand{\advB}{\mathcal{B}}
\newcommand{\advC}{\mathcal{C}}
\newcommand{\advD}{\mathcal{D}}
\newcommand{\Adv}{\mathrm{Adv}}

\newcommand{\APKE}{\mathrm{APKE}}
\renewcommand{\AE}{\mathrm{AE}}
\newcommand{\IKE}{\mathrm{IKE}}


%VARIABLES
\newcommand{\sk}{\mathit{sk}}
\newcommand{\st}{\mathit{st}}
\newcommand{\vk}{\mathit{vk}}
\newcommand{\pk}{\mathit{pk}}
\newcommand{\ek}{\mathit{ek}}
\newcommand{\dk}{\mathit{dk}}
\newcommand{\mtag}{\tau}
\newcommand{\sig}{\sigma}
\newcommand{\ad}{\mathit{ad}}

%SPACES
\newcommand{\sksp}{\mathcal{SK}}
\newcommand{\pksp}{\mathcal{PK}}
\newcommand{\vksp}{\mathcal{VK}}
\newcommand{\eksp}{\mathcal{EK}}
\newcommand{\dksp}{\mathcal{DK}}
\newcommand{\stsp}{\mathcal{ST}}
\newcommand{\ksp}{\mathcal{K}}
\newcommand{\rsp}{\mathcal{R}}
\newcommand{\msp}{\mathcal{M}}
\newcommand{\csp}{\mathcal{C}}
\newcommand{\tsp}{\mathcal{T}}
\newcommand{\sigsp}{\Sigma}
\newcommand{\adsp}{\mathcal{AD}}
\newcommand{\idsp}{\mathcal{ID}}

%ORACLES
\newcommand{\ROh}{\mathrm{H}}

\newcommand{\CORR}{\mathrm{CORR}}
\newcommand{\IND}{{\mathrm{IND}}}
\newcommand{\ind}{{\mathrm{ind}}}
\newcommand{\INDD}{{\mathrm{IND}\$}}
\newcommand{\indd}{{\mathrm{ind}\$}}
\newcommand{\INDCCA}{{\mathrm{IND}\text{-}\mathrm{CCA}}}
\newcommand{\indcca}{{\mathrm{ind}\text{-}\mathrm{cca}}}
\newcommand{\INDCPA}{{\mathrm{IND}\text{-}\mathrm{CPA}}}
\newcommand{\indcpa}{{\mathrm{ind}\text{-}\mathrm{cpa}}}
\newcommand{\OWCCA}{{\mathrm{OW}\text{-}\mathrm{CCA}}}
\newcommand{\owcca}{{\mathrm{ow}\text{-}\mathrm{cca}}}
\newcommand{\SUFCMA}{{\mathrm{SUF}\text{-}\mathrm{CMA}}}
\newcommand{\sufcma}{{\mathrm{suf}\text{-}\mathrm{cma}}}
\newcommand{\ANON}{\mathrm{ANON}}

\newcommand{\Ogen}{\mathrm{Gen}}
\newcommand{\Oinit}{\mathrm{Init}}
\newcommand{\Oenc}{\mathrm{Enc}}
\newcommand{\Odec}{\mathrm{Dec}}
\newcommand{\Oup}{\mathrm{Up}}
\newcommand{\Otag}{\mathrm{Tag}}
\newcommand{\Osig}{\mathrm{Sign}}
\newcommand{\Ovfy}{\mathrm{Vfy}}

\newcommand{\Osnd}{\mathrm{Snd}}
\newcommand{\Orcv}{\mathrm{Rcv}}
\newcommand{\Ochall}{\mathrm{Chall}}

\newcommand{\Ocorrupt}{\mathrm{Corrupt}}
\newcommand{\Oexp}{\mathrm{Expose}}
\newcommand{\Oreveal}{\mathrm{Reveal}}



\iffalse
\newcommand{\full}{\!$\cdot$\,}
\newcommand{\core}{\hphantom{\!$\cdot$\,}}

\newcommand{\getscup}{\overset{\raisebox{-1pt}{\tiny$\;\cup$}}{\gets}}
\newcommand{\getsconcat}{\overset{\raisebox{-1pt}{\tiny$\;\shortparallel$}}{\gets}} % "concat assignment"

\newcommand{\bO}{\mathcal{O}}
\newcommand{\secp}{\lambda}

\fi

%DH
\newcommand{\DHgr}{\mathbb{G}}
\newcommand{\DHg}{g}
\newcommand{\DHp}{p}

\newcommand{\dlp}{\mathrm{dlp}}
\newcommand{\cdh}{\mathrm{cdh}}
\newcommand{\ddh}{\mathrm{ddh}}

%PRG and PRF
\newcommand{\PRF}{\mathrm{PRF}}
\newcommand{\PRG}{\mathrm{PRG}}

\newcommand{\PRGf}{\PRG.\mathrm{f}}
\newcommand{\PRFf}{\PRF.\mathrm{f}}

%MAC and DIGITAL SIGNATURE
\newcommand{\MAC}{\mathrm{MAC}}
\newcommand{\SIG}{\mathrm{SIG}}

\newcommand{\MACgen}{\MAC.\mathrm{gen}}
\newcommand{\MACtag}{\MAC.\mathrm{tag}}
\newcommand{\MACvfy}{\MAC.\mathrm{vfy}}

\newcommand{\SIGgen}{\SIG.\mathrm{gen}}
\newcommand{\SIGsig}{\SIG.\mathrm{sig}}
\newcommand{\SIGvfy}{\SIG.\mathrm{vfy}}

%SYMMETRIC ENCRYPTION, PUBLIC-KEY ENCRYPTION, various KEMs, and KEY EXCHANGE
\newcommand{\SYE}{\mathrm{SE}}
\newcommand{\AEAD}{\mathrm{AEAD}}
\newcommand{\PKE}{\mathrm{PKE}}
\newcommand{\KEM}{\mathrm{KEM}}
\newcommand{\IKEM}{\mathrm{IKEM}}
\newcommand{\FKEM}{\mathrm{FKEM}}
\newcommand{\UKEM}{\mathrm{UKEM}}
\newcommand{\TKE}{\mathrm{TKE}}
\newcommand{\AKE}{\mathrm{AKE}}
\newcommand{\URKE}{\mathrm{URKE}}

\newcommand{\AEADgen}{\AEAD.\mathrm{gen}}
\newcommand{\AEADenc}{\AEAD.\mathrm{enc}}
\newcommand{\AEADdec}{\AEAD.\mathrm{dec}}

\newcommand{\SYEgen}{\SYE.\mathrm{gen}}
\newcommand{\SYEenc}{\SYE.\mathrm{enc}}
\newcommand{\SYEdec}{\SYE.\mathrm{dec}}

\newcommand{\PKEgen}{\PKE.\mathrm{gen}}
\newcommand{\PKEenc}{\PKE.\mathrm{enc}}
\newcommand{\PKEdec}{\PKE.\mathrm{dec}}

\newcommand{\KEMgen}{\KEM.\mathrm{gen}}
\newcommand{\KEMenc}{\KEM.\mathrm{enc}}
\newcommand{\KEMdec}{\KEM.\mathrm{dec}}

\newcommand{\IKEMgen}{\IKEM.\mathrm{gen}}
\newcommand{\IKEMenc}{\IKEM.\mathrm{enc}}
\newcommand{\IKEMdel}{\IKEM.\mathrm{del}}
\newcommand{\IKEMdec}{\IKEM.\mathrm{dec}}

\newcommand{\UKEMgen}{\UKEM.\mathrm{gen}}
\newcommand{\UKEMenc}{\UKEM.\mathrm{enc}}
\newcommand{\UKEMdec}{\UKEM.\mathrm{dec}}

\newcommand{\FKEMgen}{\FKEM.\mathrm{gen}}
\newcommand{\FKEMenc}{\FKEM.\mathrm{enc}}
\newcommand{\FKEMdec}{\FKEM.\mathrm{dec}}

\newcommand{\TKEsnd}{\TKE.\mathrm{snd}}
\newcommand{\TKErsp}{\TKE.\mathrm{rsp}}
\newcommand{\TKErcv}{\TKE.\mathrm{rcv}}

\newcommand{\AKEgen}{\AKE.\mathrm{gen}}
\newcommand{\AKEsnd}{\AKE.\mathrm{snd}}
\newcommand{\AKErsp}{\AKE.\mathrm{rsp}}
\newcommand{\AKErcv}{\AKE.\mathrm{rcv}}

\newcommand{\URKEinit}{\URKE.\mathrm{init}}
\newcommand{\URKEsnd}{\URKE.\mathrm{snd}}
\newcommand{\URKErcv}{\URKE.\mathrm{rcv}}

\bibliographystyle{alpha}

\makeatletter
\renewcommand*\l@author[2]{}
\renewcommand*\l@title[2]{}
\makeatletter
\renewcommand{\contentsname}{}


\title{Cryptographic Communication Protocols:\\Key Exchange and Channels}
\author{Paul Rösler}

\institute{FAU Erlangen-Nürnberg}

\begin{document}

\maketitle
\begin{center}
    \today
\end{center}

\begingroup
\let\clearpage\relax
\tableofcontents
\endgroup

\section{Preliminary Remarks}
This document is not (yet) a full course script.
Instead, it is meant as an additional resource that systematizes the considered primitives, definitions, and constructions.
The author invites readers to submit comments or pull requests via the GitHub Repository \url{https://github.com/roeslpa/ccpScript}.


\section{Notation}

\section{Game-Based Definitions}

\subsection{Syntax}

\subsection{Correctness}

\subsection{Adversarial Capabilities}

\subsubsection{Typical Capabilities}

\paragraph{Chosen-Plaintext Attacks}

\paragraph{Chosen-Ciphertext Attacks}

\paragraph{Replayable Chosen-Ciphertext Attacks}

\paragraph{Exposure of Secrets}

\paragraph{Compromised Randomness}

\subsection{Adversarial Goal}

\subsubsection{Typical Goals}

\paragraph{One-Way Security}

\paragraph{Indistinguishability of Ciphertexts}

\subsection{Trivial Winning Strategies}


\section{Symmetric Primitives}


\subsection{Pseudo-Random Generator}


\subsection{Pseudo-Random Function}


\subsection{Message Authentication Code (MAC)}


\subsection{Symmetric Encryption (SE)}

\subsubsection{Probabilistic SE}
\paragraph{Syntax}
A symmetric encryption scheme $\SYE=(\SYEgen,\SYEenc,\SYEdec)$ is a tuple of three algorithms with key space~$\ksp$, message space~$\msp$, and ciphertext space~$\csp$:

\begin{itemize}
    \item $\SYEgen: \emptyset \tor \ksp$
    \item $\SYEenc: \ksp\times\msp \tor \csp$
    \item $\SYEgen: \ksp\times\csp \to \msp$
\end{itemize}

\paragraph{Correctness}
A symmetric encryption scheme $\SYE$ is correct if $\Pr[\CORR_{\SYE}(\advA)=0]=1$ for all adversaries~$\advA$, where game~$\CORR$ is defined in Figure~\ref{fig:sym:enc:corr:prob}.

\begin{figure}[!ht]
    \centering
    \nicoresetlinenr%
    \fbox{%
        \scalebox{\codescalefactor}{%
            %!TEX root=../main.tex
\markersetlen{ndL}{100pt}%
\markersetlen{ndR}{100pt}%
\newcommand{\CM}{\mathit{CM}}%
\begin{tabular}[t]{ll}
    \nicodemusbox{\markerlenndL}{%
        \textbf{Game} $\CORR_{\SYE}(\advA)$
        \begin{nicodemus}
            \item $\CM[\cdot]\gets\bot$
            \item $k\getsr\SYEgen$
            \item Invoke $\advA$
            \item Stop with~$0$
        \end{nicodemus}%
        \medskip
        
        \textbf{Oracle} $\Oenc(m)$
        \begin{nicodemus}
            \item Require $m\in\msp$
            \item $c\getsr\SYEenc(k,m)$
            \item $\CM[c]\gets m$
            \item Return~$c$
        \end{nicodemus}%
    }%
    &
    \nicodemusbox{\markerlenndR}{%
        \textbf{Oracle} $\Odec(c)$
        \begin{nicodemus}
            \item $m'\gets\SYEdec(k,c)$
            \item If $\CM[c]\notin\{m',\bot\}$:
            \item \quad Stop with $1$
            \item Return $m'$
        \end{nicodemus}%
    }%
\end{tabular}%%
        }%
    }
    \caption{%
        Game $\CORR$ for probabilistic encryption scheme~$\SYE$.
    }
    \label{fig:sym:enc:corr:prob}
\end{figure}

\paragraph{Security: One-Wayness under Chosen-Ciphertext Attacks}
The advantage of an adversary~$\advA$ against symmetric encryption scheme $\SYE$ in game $\OWCCA$ from Figure~\ref{fig:sym:enc:corr:prob} is defined as:\\
$\Adv_\SYE^\owcca(\advA)\coloneqq\Pr[\OWCCA_{\SYE}(\advA)=1]$.

\begin{figure}[!ht]
    \centering
    \nicoresetlinenr%
    \fbox{%
        \scalebox{\codescalefactor}{%
            %!TEX root=../main.tex
\markersetlen{ndL}{100pt}%
\markersetlen{ndR}{130pt}%
\newcommand{\CM}{\mathit{CM}}%
\begin{tabular}[t]{ll}
    \nicodemusbox{\markerlenndL}{%
        \textbf{Game} $\OWCCA_{\SYE}(\advA)$
        \begin{nicodemus}
            \item $\CM\gets\emptyset$
            \item $k\getsr\SYEgen$
            \item $(c,m)\getsr\advA$
            \item If $(c,m)\in\CM$:
            \item \quad Stop with~$1$
            \item Stop with~$0$
        \end{nicodemus}%
        \medskip
        
        \textbf{Oracle} $\Oenc(m)$
        \begin{nicodemus}
            \item $c\getsr\SYEenc(k,m)$
            \item Return~$c$
        \end{nicodemus}%
    }%
    &
    \nicodemusbox{\markerlenndR}{%
        \textbf{Oracle} $\Ochall()$
        \begin{nicodemus}
            \item $m\getsr\msp$
            \item $c\getsr\SYEenc(k,m)$
            \item $\CM\gets\CM\cup\{(c,m)\}$
            \item Return~$c$
        \end{nicodemus}%
        \medskip

        \textbf{Oracle} $\Odec(c)$
        \begin{nicodemus}
            \item Require $\nexists m':(c,m')\in\CM$
            \item $m\gets\SYEdec(k,c)$
            \item Return $m$
        \end{nicodemus}%
    }%
\end{tabular}%%
        }%
    }
    \caption{%
        Game $\OWCCA$ for probabilistic symmetric encryption scheme~$\SYE$.
    }
    \label{fig:sym:enc:ow:prob}
\end{figure}

\paragraph{Security: Indistinguishability under Chosen-Ciphertext Attacks}
The advantage of an adversary~$\advA$ against symmetric encryption scheme $\SYE$ in game $\INDCCA$ from Figure~\ref{fig:sym:enc:ind:prob} is defined as:\\
$\Adv_\SYE^\indcca(\advA)\coloneqq\left|\Pr[\INDCCA_{\SYE}^0(\advA)=1]-\Pr[\INDCCA_{\SYE}^1(\advA)=1]\right|$.

\begin{figure}[!ht]
    \centering
    \nicoresetlinenr%
    \fbox{%
        \scalebox{\codescalefactor}{%
            %!TEX root=../main.tex
\markersetlen{ndL}{100pt}%
\markersetlen{ndR}{130pt}%
\newcommand{\CC}{\mathit{CC}}%
\begin{tabular}[t]{ll}
    \nicodemusbox{\markerlenndL}{%
        \textbf{Game} $\INDCCA_{\SYE}^b(\advA)$
        \begin{nicodemus}
            \item $\CC\gets\emptyset$
            \item $k\getsr\SYEgen$
            \item $b'\getsr\advA$
            \item Stop with~$b'$
        \end{nicodemus}%
        \medskip
        
        \textbf{Oracle} $\Oenc(m)$
        \begin{nicodemus}
            \item $c\getsr\SYEenc(k,m)$
            \item Return~$c$
        \end{nicodemus}%
    }%
    &
    \nicodemusbox{\markerlenndR}{%
        \textbf{Oracle} $\Ochall(m_0,m_1)$
        \begin{nicodemus}
            \item Require $\{m_0,m_1\}\subseteq\msp$
            \item $c\getsr\SYEenc(k,m_b)$
            \item $\CC\gets\CC\cup\{c\}$
            \item Return~$c$
        \end{nicodemus}%
        \medskip
        
        \textbf{Oracle} $\Odec(c)$
        \begin{nicodemus}
            \item Require $c\notin\CC$
            \item $m\gets\SYEdec(k,c)$
            \item Return $m$
        \end{nicodemus}%
    }%
\end{tabular}%%
        }%
    }
    \caption{%
        Games $\INDCCA$ for probabilistic symmetric encryption scheme~$\SYE$.
    }
    \label{fig:sym:enc:ind:prob}
\end{figure}


\subsubsection{Nonce-Based SE}


\subsubsection{Authenticated Encryption (AE)}



\section{Asymmetric Primitives}


\subsection{Assumptions}

\subsubsection{Diffie-Hellman (DH)}

\paragraph{Discrete Logarithm}

\paragraph{Computational DH}

\paragraph{Decisional DH}


\subsection{Digital Signature}


\subsection{Public-Key Encryption (PKE)}


\subsection{Key Encapsulation Mechanism (KEM)}

\paragraph{Construction: ElGamal KEM}


\subsection{Identity-Based KEM}



\section{Stateful Primitives}

\subsection{Forward-Secure KEM}

\paragraph{Construction}


\subsection{Updatable KEM}

\paragraph{Construction}


\subsection{Key Exchange}

\subsubsection{Two-Pass Key Exchange}

\paragraph{Construction: DH Key Exchange}

\paragraph{Construction: X3DH}

\subsubsection{Authenticated Key Exchange}

\paragraph{Construction: Signature-Based Authentication}

\paragraph{Construction: KEM-Based Authentication}

\paragraph{Construction: TLS}

\paragraph{Construction: Noise Framework}


\subsection{Ratcheted Key Exchange (RKE)}

\subsubsection{Unidirectional RKE}

\paragraph{Construction}

\subsubsection{Sesquidirectional RKE}

\subsubsection{Bidirectional RKE}

\paragraph{Construction: Double Ratchet}

\subsubsection{Group RKE}

\paragraph{Construction: Sender Key Mechanism}

\paragraph{Construction: Tree-Based DH}

\paragraph{Construction: Tree KEM}

\bibliography{cryptobib/abbrev3,cryptobib/crypto}

\end{document}